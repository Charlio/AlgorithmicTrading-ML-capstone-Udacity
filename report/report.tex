% !TEX TS-program = pdflatex
% !TEX encoding = UTF-8 Unicode

% This is a simple template for a LaTeX document using the "article" class.
% See "book", "report", "letter" for other types of document.

\documentclass[11pt]{article} % use larger type; default would be 10pt

\usepackage[utf8]{inputenc} % set input encoding (not needed with XeLaTeX)

%%% Examples of Article customizations
% These packages are optional, depending whether you want the features they provide.
% See the LaTeX Companion or other references for full information.

%%% PAGE DIMENSIONS
\usepackage{geometry} % to change the page dimensions
\geometry{a4paper} % or letterpaper (US) or a5paper or....
% \geometry{margin=2in} % for example, change the margins to 2 inches all round
% \geometry{landscape} % set up the page for landscape
%   read geometry.pdf for detailed page layout information

\usepackage{graphicx} % support the \includegraphics command and options

% \usepackage[parfill]{parskip} % Activate to begin paragraphs with an empty line rather than an indent

%%% PACKAGES
\usepackage{booktabs} % for much better looking tables
\usepackage{array} % for better arrays (eg matrices) in maths
\usepackage{paralist} % very flexible & customisable lists (eg. enumerate/itemize, etc.)
\usepackage{verbatim} % adds environment for commenting out blocks of text & for better verbatim
\usepackage{subfig} % make it possible to include more than one captioned figure/table in a single float
% These packages are all incorporated in the memoir class to one degree or another...

%%% HEADERS & FOOTERS
\usepackage{fancyhdr} % This should be set AFTER setting up the page geometry
\pagestyle{fancy} % options: empty , plain , fancy
\renewcommand{\headrulewidth}{0pt} % customise the layout...
\lhead{}\chead{}\rhead{}
\lfoot{}\cfoot{\thepage}\rfoot{}

%%% SECTION TITLE APPEARANCE
\usepackage{sectsty}
\allsectionsfont{\sffamily\mdseries\upshape} % (See the fntguide.pdf for font help)
% (This matches ConTeXt defaults)

%%% ToC (table of contents) APPEARANCE
\usepackage[nottoc,notlof,notlot]{tocbibind} % Put the bibliography in the ToC
\usepackage[titles,subfigure]{tocloft} % Alter the style of the Table of Contents
\renewcommand{\cftsecfont}{\rmfamily\mdseries\upshape}
\renewcommand{\cftsecpagefont}{\rmfamily\mdseries\upshape} % No bold!

%%% END Article customizations

%%% The "real" document content comes below...

\title{Algorithmic Trading with Machine Learning and Deep Learning}
\author{Charlio Xu}
%\date{} % Activate to display a given date or no date (if empty),
         % otherwise the current date is printed 

\begin{document}
\maketitle

\section{Introduction}
This project sets up an algorithmic trading platform which incorporates various kinds of machine learning and deep learning methods to learn and predict market movement directions in order to generate extra cumulative returns over the benchmark returns.

Compare algorithmic trading with machine learning: different final stages. Prediction and transaction instead of testing


\section{Domain Background}

a. automatic trading of equities with predefined algorithms, order placement is determined by algorithms 

b. prediction of market movement directions vs price prediction, then place long or short order

c. classification problem with labels of two categories

d. machine learning and deep learning algorithms can be used




\section{Problem Statement}

given a history of past prices of a stock or other equities, how can we predict its movement directions (upwards or downwards) in the future 


\section{Datasets and Inputs}

a. historical daily closed prices of stocks from Yahoo, or high-frequency tick data

b. input will be time series price data of a stock: time-indexed daily closed prices (float)


\section{Solution Statement}

The problem is a classification problem, so I will use supervised learning of classification like logistic regression, SVM, or dicision tree, and deep learning methods like simple neural network or convolutional neural network to tackle the problem.


\section{Benchmark Model}

I will use classical simple trading strategies as benchmark models, which include simple moving average, momentum and mean reversion methods. 

Moreover, I will also compare the machine learning and deep learning methods. 


\section{Evaluation Metrics}

On the machine learning level, I will use accuracy for our classification problem.

On the algorithmic trading level, I will use the cumulative returns including transaction costs as the evluation metric and compare it with the realized return in a plot.


\section{Project Design}

For the core algorithms, I will use classification algorithms in scikit-learn as well as construct neural networks in TensorFlow or Keras.

For the learning process, I will first use vectorized backtesting in numpy to have a first look at the performance of an algorithm, and then use event-based backtesting with object-oriented programming paradign to enhance the learning process.

Data will be downloaded from Yahoo! Finance and preprocessed with Pandas. Since the data is time series, spliting it into training, validating and testing sets is easy by simply cutting the time line into three pieces.

Evaluation will be provided with an accuracy analysis, but more trading-oriented performance will be visualized in a plot to compare the cumulative returns between given by the algorithm and the realized historic return.




\subsection{Vectorized Backtesting}










\end{document}
