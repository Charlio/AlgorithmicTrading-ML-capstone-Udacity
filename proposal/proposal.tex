% !TEX TS-program = pdflatex
% !TEX encoding = UTF-8 Unicode

% This is a simple template for a LaTeX document using the "article" class.
% See "book", "report", "letter" for other types of document.

\documentclass[11pt]{article} % use larger type; default would be 10pt

\usepackage[utf8]{inputenc} % set input encoding (not needed with XeLaTeX)

%%% Examples of Article customizations
% These packages are optional, depending whether you want the features they provide.
% See the LaTeX Companion or other references for full information.

%%% PAGE DIMENSIONS
\usepackage{geometry} % to change the page dimensions
\geometry{a4paper} % or letterpaper (US) or a5paper or....
% \geometry{margin=2in} % for example, change the margins to 2 inches all round
% \geometry{landscape} % set up the page for landscape
%   read geometry.pdf for detailed page layout information

\usepackage{graphicx} % support the \includegraphics command and options

% \usepackage[parfill]{parskip} % Activate to begin paragraphs with an empty line rather than an indent

%%% PACKAGES
\usepackage{booktabs} % for much better looking tables
\usepackage{array} % for better arrays (eg matrices) in maths
\usepackage{paralist} % very flexible & customisable lists (eg. enumerate/itemize, etc.)
\usepackage{verbatim} % adds environment for commenting out blocks of text & for better verbatim
\usepackage{subfig} % make it possible to include more than one captioned figure/table in a single float
% These packages are all incorporated in the memoir class to one degree or another...

%%% HEADERS & FOOTERS
\usepackage{fancyhdr} % This should be set AFTER setting up the page geometry
\pagestyle{fancy} % options: empty , plain , fancy
\renewcommand{\headrulewidth}{0pt} % customise the layout...
\lhead{}\chead{}\rhead{}
\lfoot{}\cfoot{\thepage}\rfoot{}

%%% SECTION TITLE APPEARANCE
\usepackage{sectsty}
\allsectionsfont{\sffamily\mdseries\upshape} % (See the fntguide.pdf for font help)
% (This matches ConTeXt defaults)

%%% ToC (table of contents) APPEARANCE
\usepackage[nottoc,notlof,notlot]{tocbibind} % Put the bibliography in the ToC
\usepackage[titles,subfigure]{tocloft} % Alter the style of the Table of Contents
\renewcommand{\cftsecfont}{\rmfamily\mdseries\upshape}
\renewcommand{\cftsecpagefont}{\rmfamily\mdseries\upshape} % No bold!

%%% END Article customizations

%%% The "real" document content comes below...

\title{Algorithmic Trading with Machine Learning}
\author{Chao (Charlio) Xu}
%\date{} % Activate to display a given date or no date (if empty),
         % otherwise the current date is printed 

\begin{document}
\maketitle

\section{Introduction}
algorithmic trading of equities in Python with machine learning and deep learning algorithms to predict market movement directions and equity prices in order to generate extra cumulative returns over the benchmark returns


\section{Domain Background}
the field of research where the project is derived: automatic trading of equities with predefined algorithms 


\section{Problem Statement}
a problem being investigated for which a solution will be defined: how to predict the market movement directions, and further the equity prices


\section{Datasets and Inputs}
data or inputs being used for the problem: historic prices of equities from Quandl or Oanda, real-time streaming price data for prediction and testing


\section{Solution Statement}
a the solution proposed for the problem given: supervised classification learning to predict market movement directions, and supervised regression learning to predict equity prices; further to use neural networks to enhance the learning 


\section{Benchmark Model}
some simple or historical model or result to compare the defined solution to: simple moving average, momentum


\section{Evaluation Metrics}
functional representations for how the solution can be measured: cumulative returns excluding transaction costs or not


\section{Project Design}
how the solution will be developed and results obtained: develop in Python; pandas to download financial data; scikit-learn for machine learning algorithms; TensorFlow or Keras for neural networks deep learning


\end{document}
